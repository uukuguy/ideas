% !Mode:: "TeX:UTF-8"

由定义 \ref{def1}可知,词的共现率是两个词在同一主题内同时出现的概率之和,然而在实际文本建模过程中, 主题是文本的隐含变量,无法准确获得。因此,通过统计两个词在文本中共现的次数计算它们的共现率。假设一个窗口单元(例如,一句话、一个段落)代表一个主题,文本空间中有$n$个窗口单元,则主题的先验概率为$P(t) = 1 / n$。当词$w$在窗口单元出现,则其后验概率为$P(w \mid t) = 1$。根据公式
\ref{formula1},如果词$w_i$和$w_j$共同出现在$x$个不同的窗口单元中,它们的联合概率为$P(w_i, w_j)= x / n$。因此词的共现率可以由公式 \ref{formula2}计算得出。

\begin{equation}
P(w_i, w_j) = \frac{\parallel Segmgnet(w_i, w_j) \parallel}{\parallel Segment \parallel}
\label{formula2}
\end{equation} 

式中,$Segment(w_i, w_j)$表示文本空间中同时包含$w_i$和$w_j$的窗口单元集合,$Segment$表示文本空间中的窗口单元集合,$\parallel \cdot \parallel$表示集合中元素的个数。例如,将窗口单元设为一个自然段,则自然段内出现的词汇对视为它们的一次共现。
