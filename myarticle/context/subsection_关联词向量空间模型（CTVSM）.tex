% !Mode:: "TeX:UTF-8"

与VSM的建模思想类似,CTVSM将文本表示为一个共现词组合的向量。设文档空间$D = \{d_1,d_2, \ldots,d_n\}$中包含$n$篇文档,在$D$上抽取出的共现词组合的集合为$C=\{c_1,c_2, \ldots,c_m \}$,其中$c_m$ 为抽取出的第$m$个共现词组合。则文档空间$D$可表示为一个$m \times n$的矩阵。

\begin{equation}
D = (d_1, d_2, \ldots, d_n)^T
\end{equation}

其中行向量$d_i=(c_{i1},c_{i2}, \ldots,c_{im} )$代表一篇文档,


\begin{equation}
d_i = (c_{i1}, c_{i2}, \ldots, c_{im})
\end{equation}

列向量$c_j = (c_{1j}, c_{2j}, \ldots,c_{nj})$代表一个共现词汇组合在各文档中 的分布情况。矩阵中的元素$c_{ij}$表示文档$d_i$中共现词汇组合$c_j$的分布情况。如果共现词汇组合出现则相应的权值为1,如果不出现,则相应权值为0。

%\begin{equation}
%c_{ij} = 
%\begin{cases} 
    %0 & c\_j \notin d\_i  \\\\
    %1 & c\_j \in d\_i
%\end{cases}
%\end{equation}


