% !Mode:: "TeX:UTF-8"

词(Term)是文本中包含语义的最小组成单元,文本向量由文本中的词构成。不同词性的词在语义表达上的作用是不同的,虚词(副词、介词、连词等)是没有完整意义的词汇,只有语法意义或功能意义,不作为单独的语法成分。实词有实在意义,能够单独充当句子成分,一般能单独回答问题,包括名词、动词、形容词、数词、量词、代词六类。因此,本研究中文本关键词的选择由文本中的名词、动词、动名词构成。\citet{kottwitz2011latex}提出了一种算法,\citet{NI-OVERSAMPLING}提出了另一种算法。

\begin{equation}
D_i = ( w_{i,1}, w_{i,2}, \ldots w_{i,n} )
\end{equation}

上式是原生文本向量,即常规的向量空间模型,其中$D_i$是语料库中第$i$个文本向量,$w_{i,n}$是文本$D_i$中第$n$个关键词(仅选择名词、动词、动名词)。

\begin{equation}
Corpus = \{ d_1, d_2, \ldots d_n \}
\end{equation}

以上是语料库模型,语料库Corpus由语料库中所有文本向量d$_n$的集合。
