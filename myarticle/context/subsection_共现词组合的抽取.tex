% !Mode:: "TeX:UTF-8"

共现词组合的抽取过程类似于利用关联规则的数据挖掘方法发现事务中频繁项集的过程。通过关联规则挖掘算法,挖掘语料库中词与词之间的关系 ,其中支持度由公式 \ref{formula_support0}表示,置信度由公式 \ref{formula_identify0}表示。

\begin{equation}
Support(w_i, w_j) = P(w_i, w_j)
\label{formula_support0}
\end{equation}

\begin{equation}
Identify(w_i, w_j) = \frac{1}{2}\left( \frac{P(w_i, w_j)}{P(w_i)} + \frac{P(w_i, w_j)}{P(w_j)} \right)
\label{formula_identify0}
\end{equation}

给定关联规则挖掘空间$S = (T, W, R, \theta)$,其中各项含义如下:

1. $T=\{t_1, t_2, \ldots, t_m\}$为空间$S$上的文本段集合(Text Segment Set),其中$t_m \in T$为空间$S$上的文本段,即文本中的句子、自然段或全文;
2. $W=\{ w_1, w_2, \ldots, w_n \}$为空间$S$上的词汇集(Word Set),其中$w_n \in W$为空间$S$上的候选词汇;
3. $R=\{ r_1, r_2, \ldots, r_k \}$为文本段集合$T$中蕴含的规则,其中$r_k=(w_x, w_y), w_x, w_y \in W$,即抽取出的共现词组合;
4. $\theta=\{ \alpha, \beta \}$为空间$S$上的约束,$\alpha$和$\beta$分别为给定的支持度与置信度的阈值。

在空间$S$上的“文本段-词汇”矩阵可以表示为$m \times n$的矩阵。其中行向量代表一个文本段$t$(句子、自然段或全文),列向量代表词汇$w$(候选词汇),矩阵为0-1矩阵。

空间$S$上的支持度由公式 $\ref{formula_support}$表示,置信度由公式 $\ref{formula_identify}$表示。

\begin{equation}
support(w_x, w_y) = \sum\limits_{k=1}^{m} (w_{kx} \times w_{ky}) 
\label{formula_support} 
\end{equation}

\begin{equation}
identify(w_x, w_y) = \frac{1}{2}\left [ 
\frac{\sum\limits_{k=1}^{m}w_{kx} \times w_{ky}}{\sum w_x} +
\frac{\sum\limits_{k=1}^{m}w_{kx} \times w_{ky}}{\sum w_y} 
\right ]
\label{formula_identify}
\end{equation}

在上述定义的关联规则挖掘空间$S$上,有以下挖掘算法:

\begin{algorithm}[htb]
\caption{利用关联规则挖掘共现词组合算法}
\label{alg:co_word}
\begin{algorithmic}[1]
\REQUIRE 文本语料库$Corpus$,阈值$\theta$
\ENSURE 共现词组合列表$R$
\STATE 读取语料库$Corpus$,分词后生成$m \times n$的“文本段-词汇”矩阵$\{ P, W\}$;
\STATE 生成所有二阶词对组合$(w_x, w_y), \forall w_x, w_y \in W$;
\FOR{ each $(w_x, w_y)$}
    \STATE // 按公式 \ref{formula_support}计算支持度$s$
    \STATE $support(w_x, w_y) \to s$
    \STATE // 按公式 \ref{formula_identify}计算置信度$i$ 
    \STATE $identify(w_x, w_y) \to i$     
    \STATE // 如果$(s,i)$大于设定阈值$\theta$,则将该组合加入$R$
    \IF {$s \geq \alpha \wedge i \geq \beta$}
        \STATE $(w_x, w_y) \to R$
    \ENDIF
\ENDFOR
\RETURN $R$
\end{algorithmic}
\end{algorithm}

此算法只在开始时读取一次语料库,在内存中构建矩阵,之后利用矩阵运算求解词共现的支持度和置信度。运行效率较高,且算法的复杂度为$O(n^2)$。

