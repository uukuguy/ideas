% !Mode:: "TeX:UTF-8"

通常,文本中出现的词汇都是为了帮助作者表达其主题思想,只不过每个词与主题的相关程度不同。词汇有其所属主题域,相同主题域中的词汇共同出现在同一篇文档中的概率相对较高,因此利用词的共现现象可以判断词汇与主题之间的相关程度。

假设文本空间$D$上有主题集合$T$和词汇集合$W$,其中,d$_i$ $\in D$代表第$i$篇文档,t$_k$ $\in T$代表文档空间中的第$k$个主题,$S(t_k)$为主题$t_k$的主题相关词集合,$w_m \in W$为文档中出现的词汇。在主题$t_k$出现的情况下,词汇$w_m$出现的条件概率可表示为$P(w_m \mid t_k)$。

\begin{definition}[共现率]
词汇$w_i$与$w_j$的共现率是指这两个词在同一文本空间单位中共同出现的概率,即它们在文本空间中的联合概率, \label{def1}如下式所示:
\begin{equation}
P(w_i, w_j) = \sum_{i \in T}P(t) P(w_i \mid t)P(w_j \mid t), \forall w_i, w_j \in W
\label{formula1}
\end{equation} 
\end{definition}


\begin{definition}[显著性事件]
设满足$P(\cdot) > \theta$的事件,即发生概率较高的事件,被称为显著性事件。其中$\theta$为显著性判别标准,通常与语料库的规模以及主题在文本空间中的分布有关。 \label{def2}
\end{definition}

\begin{definition}[主题相关]
当$P(w_m \mid t_k) \geq \theta$,即是显著性事件时,$w_m$是$t_k$的主题相关词汇,称为$w_m$与$t_k$主题相关。
\end{definition} \label{def3}

\begin{definition}[主题无关]
当$P(w_m \mid t_k) < \theta$,即是非显著事件时,$w_m$不是$t_k$的主题相关词汇,称$w_m$与$t_k$主题无关。 \label{def4}
\end{definition}

\begin{corollary}
如果两个词汇的共现为显著事件,则这两个词汇与某个共同的主题相关。即:当$P(w_i, w_j) \geq \theta$时,$\exists t_k$ 使得 $P(w_i \mid t_k) \geq \theta$ 且 $P(w_j \mid t_k) \geq \theta$。 \label{coro1}
\end{corollary}

由此推论 \ref{coro1}可知,如果两个词汇的共现率超过一定的阈值,则预示着这两个词较大可能是主题相关的。
