% !Mode:: "TeX:UTF-8"

事件e被定义为6元组格式:

\begin{equation}
e = (A, O, T, V, P, L)
\end{equation}

A是事件中的动作,O是事件中的对象,T是事件发生的时间,V是事件发生的环境(包括自然环境和社会环境),P是事件中动作执行过程的断言,L是语言表达式。

事件触发词(Event Trigger Word),统计上表明词性主要是名词、动词、动名词。事件触发词窗口采用前3后2共6个词,第4个是事件触发词。

\begin{itemize}
\tightlist
\item
  \href{http://blog.csdn.net/shijiebei2009/article/details/44538257}
  {事件本体以及突发事件语料库--CEC(Chinese
  Emergency Corpus)}
\item
  \href{https://github.com/shijiebei2009/CEC-Corpus}
  {中文突发事件语料库}
\item
  \href{https://github.com/shijiebei2009/CEEC-Corpus}
  {中文环境突发事件语料库}
\item
  \href{https://github.com/shijiebei2009/CEC-Automatic-Annotation}
  {基于CEC语料库挖掘要素识别规则,对新闻报道类生语料进行自动标注}
\end{itemize}

