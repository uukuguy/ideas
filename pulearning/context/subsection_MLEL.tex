% !Mode:: "TeX:UTF-8"

\citet{AAAI113583}提出MLEL算法\cite{AAAI113583},首先使用一个启发式方法(Heuristic Method)按可信度(Confidence)生成多级样本(Multi-level examples),样本的数量和质量对训练一个高质量的分类器非常重要。难以同时保证准确率和召回率,必须在两者之间选择权衡。其次,使用加权支持向量机(WSVM)来不同对待多级训练样本。共分为五级:$GP$(Golden Positives)、$PP$(Potential Positives)、$SN$(String Negatives)、$RN$(Reliable Negatives)、$PN$(Potential
Negatives)。正例度$PD$(Positive Degree)被用来裁决一个未标注样本是否正例样本。具体见算法 \ref{alg:MLEL}。

\begin{algorithm}[htb]
\caption{MLEL($P,U$)}
\label{alg:MLEL}
\begin{algorithmic}[1]
\REQUIRE positive documents $P$, unlabeled document $U$
\ENSURE a text classifier
\STATE Obtain positive feature set ($PF$) and word positive
\STATE degree($PD_{word}$) for each feature using Positive Feature
\STATE Selection algorithm.
\STATE Use Multi-level Example Generation algorithm to
\STATE obtain $GP, PP, SN, RN$ and $PN$.
\STATE Train text classifier using WSVM.
\RETURN 
\end{algorithmic}
\end{algorithm}

\subsubsection{Positive Feature Selection}
正例特征使用可以表达正例样本并能与负例样本区分开的词条。用两个统计标准量$Specialty$和$Popularity$来度量一个词条是否正例特征。
当一个词条在正例样本集合$P$中出现的频率超过在混合集合中出的频率时,该词条更倾向地是一个正例特征。如下公式,当词条的$Specialty > 0.5$时,更倾向于是一个正例特征。

\begin{equation}
Specialty(w) = f(w,P)/(f(w,P) + f(w,U))
\end{equation}

假设集合$P$中两个词条具有相同的出现频率,其中在集合$P$中更多样本中出现的那个词条,相对另一个更倾向于正例特征。用信息熵$Ent(w,P)$来描述集合P中词条$w$的分布情况。

\begin{equation}
Popularity(w) = Ent(w,P) / Z
\end{equation}

$Z$是信息熵$Ent(w,P)$的最大值,可以取$log(n_p)$,$n_p$是集合P中样本总数。

\begin{equation}
Ent(w,P) = -\sum\limits_{i=1}^{n_p}NProb(d_i \mid w)log(NProb(d_i \mid w))
\end{equation}

\begin{equation}
NProb(d_i \mid w) = \frac{Prob(d_i \mid w) / l_i}{\sum\limits_{j=1}^{n_p}Prob(d_j \mid w) / l_j} 
\end{equation}

\begin{equation}
Prob(d_i \mid w) = f(w,d_i)/f(w,P) 
\end{equation}

\begin{equation}
l_i = \sum\limits_{w \in d_i}f(w,d_i)
\end{equation}

\begin{equation}
PD_{word}(w) = Specialty(w) + Popularity(w)
\end{equation}

判别条件满足$Popularity(w)$ > $\alpha$,$Specialty(w)$ > $\beta$,$PD_{word}(w)$ > $\gamma$时,词条w是一个正例特征。$\alpha$、$\beta$、$\gamma$分别是$Popularity$、$Specialty$、$PD_{word}$的判定阈值,经验值可以取实验中的平均值。

\subsubsection{Multi-Level Example Generation}

\begin{itemize}
\item Document Positive Degree  

样本正例度越大,该样本越倾向于是一个正例样本。

\begin{equation}
PD_{doc}(d_i) = \frac{\sum\limits_{w \in PF, w \in d_i}PD_{word}(w)}{log(l_i)}
\end{equation}

\item Multi-Level Positives Acquisition

已有的正例训练样本集合P中的样本被称为$GP$(Golden Positives)。
样本$PD_{doc}$大于集合P中所有样本$PD_{doc}$的平均值的,被称为$PP$(Potential Positives)。

\begin{equation}
PD_{doc}(d_x) > \overline{PD(P)} 
\end{equation}

\item Multi-Level Negatives Acquisition

样本$PD_{doc}$ = 0,被称为$SN$(Strong Negatives)。通常是不包含任何正向特征的样本。
样本$PD_{doc}$小等于集合U中所有样本$PD_{doc}$的平均值的,被称为$RN$(Reliable Negatives)。

\begin{equation}
0 < PD_{doc}(d_x) \leq \overline{PD(U)}
\end{equation}

集合M中刨除$PP$、$SN$、$RN$剩余的样本被称为$PN$(Potential Negatives)。经验表明,PN对训练分类器同样有用。

\end{itemize}

\subsubsection{Multi-Level Example Based Learning}

使用加权支持向量机WSVM基于多级样本训练分类器,给样本分别赋予不同的权值。

最小化: $\frac{1}{2}\left \| W \right \| ^2 + 
        c_{+}^{'}\sum_{i \in GP}\xi_i +
        c_{+}^{''}\sum_{i \in PP}\xi_i +
        c_{-}^{'}\sum_{i \in SN}\xi_i +
        c_{-}^{''}\sum_{i \in RN}\xi_i +
        c_{-}^{'''}\sum_{i \in PN}\xi_i 
        $

优化目标: $y_i(w^Tx_i + b) \geq 1 - \xi_i (i = 1,2, \ldots, n)$

$\xi_i$是松弛变量(slack variable),用于允许误分类部分训练样本。$c_{+}$$c_{-}$等变量分别是$GP$、$PP$、$SN$、$RN$、$PN$的惩罚因子(Penalty Factor),用于调整不同级别样本可信度造成的影响。通常正例样本数量远远低于 未标注样本集合中负例样本数量。


