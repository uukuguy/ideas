% !Mode:: "TeX:UTF-8"

当前主要的相关解决算法包括S-EM\cite{liu2002partially}、Roc-SVM\cite{li2003learning}、Biased-SVM\cite{1250918}、CR-SVM\cite{li2010negative}、LELC\cite{li2009positive}、MLEL\cite{AAAI113583}、SPUL\cite{xiao2011similarity}、auto-CiKL\cite{hu2012estimate}。其中S-EM、Roc-SVM、CR-SVM、PEBL等都 是采用两步骤策略:1) 自动从未标注数据集中识别负例样本,以扩大负例训练样本集合;2)用已有的正例训练样本和新识别出的负例训练样本来构建分类器。该方法的缺陷在于认为所有训练样本都具有同等的可信度。事实上,由于 已有的正例训练样本是人工标注的,是高可信的黄金数据(Golden Data),而自动识别出的负例训练样本由于 不可避免的分类错误,可信度较低。Biased-SVM和MLEL在这方面都有相应的考虑和调整,Biased-SVM在正例训练样本较少时分类性能较差\cite{AAAI113583}。

\begin{itemize}
\item Biased-SVM\cite{1250918}

\citet{1250918}综述了S-EM、PEBL、Roc-SVM,并提出新算法\cite{1250918}。将未标注集合M中的所有样本初始标注为负例,并以不同分类权重构建SVM。负例分类权重实验值0.01,0.03.0.05, ....,0.61,正例分类权重相对负例分类权重的倍数实验中取10,20,30, ..., 200。

\item LELC\cite{li2009positive}

\citet{li2009positive}首先用Spy-EM和Rocchio两种 方法找出可信负例样本交集NS,构建典型正例和负例原型,其次将模糊样本(Ambiguous Examples)聚类成微聚类(micro-cluster),计算微聚类与正例原型和负例原型之间的余弦相似度,用Rocchio分类器判别微聚类为近似正例样本(likely positive examples)或近似负例样本(likely nagetive examples),生成新的正例样本集合(原正例样本+近似正例样本)和负例样本集合(可信负例样本+近似负例样本),最后构建最终的SVM分类器。

\item MLEL\cite{AAAI113583}

\citet{AAAI113583}提出了正例度(Positive Degree)的概念及算法\cite{AAAI113583},可以计算出词条正例度(PDword)和文本正例度(PDdoc)。基于正例度已能较显著地(目测方式)从未标注集合中获得大量正例样本。根据文本正例度大小范围,将未标注集合中的样本分为多个级别的正例和负例样本,用WSVM构建分类器。

\item SPUL\cite{xiao2011similarity} 

\citet{xiao2011similarity}针对模糊样本(Ambiguous Examples. 不能明确判定为正例或负例的样本)提出相似度指标(similarity)。首先用与LELC相同的方法抽取可信负例样本集NS,并构建典型正例和负例原型,其次通过相似度权值计算方法(本地相似度权值和全局相似度权值)计算每个未标注样本的正例相似度和负例相似度,最后基于正例相本集PS,负例样本集NS以及模糊样本集US构建SVM分类器。

\item auto-CiKL\cite{hu2012estimate}

\citet{hu2012estimate}针对正例样本在集合P和集合M中不同的分布密度,造成各种 已有算法性能随之不同的现象,提出了估算未标注集合M中负例样本比例的算法,并根据估算的比例选择不同的算法构建分类器。



\end{itemize}

