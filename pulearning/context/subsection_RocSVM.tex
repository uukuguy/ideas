% !Mode:: "TeX:UTF-8"

\citet{li2003learning}提出了算法\cite{li2003learning},在 Rocchio分类器中,构造一个原型向量来总结类$j$中的文档,每个待分类文档需要与每个类的原型向量计算相似度,这个待分类的文档很有可能属于相似度最大的类。原型向量构造见式 \ref{protocal_vector}

\begin{equation}
\overrightarrow{c_j} = \alpha \frac{1}{\left | c_j \right |}
         \sum\limits_{\overrightarrow{d} \in c_j}\frac{\overrightarrow{d}}{\left \| \overrightarrow{d} \right \| } -
         \beta \frac{1}{\left | D - c_j \right |}
         \sum\limits_{\overrightarrow{d} \in D - c_j} 
         \frac{\overrightarrow{d}}{\left \| \overrightarrow{d} \right \|}
\label{protocal_vector}
\end{equation}

其中,$\left \| \overrightarrow{d} \right \|$代表向量$\overrightarrow{d}$的欧几里得距离,$D$代表训练集,$c_j$是识别出的相关文档的集合,$\alpha$和$\beta$这两个参数用来调整识别相关文档和不相关文档的效果,试验表明两个参数的合理取值$\alpha$和$\beta$分别设置为16和4。

paragraph{从未标注样本集$M$中识别尽可能多包含负例样本的可信负例样本集$RN$}

paragraph{在可信正例样本集$P$和可信负例样本集$RN$上迭代构建分类器,从中选择最佳分类器}

