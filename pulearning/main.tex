% !Mode:: "TeX:UTF-8"

\documentclass[cs5size,oneside,twocolumn]{ctexart}

\usepackage{ifthen}
\usepackage{setspace}


\setmainfont{Times New Roman}

%%%%%%%%%% page margin %%%%%%%%%%
% 页面边距

\usepackage{geometry}
\newgeometry{
    top=25mm, bottom=25mm, left=20mm, right=20mm,
    headsep=5mm, headheight=10mm, footskip=10mm,
}
\savegeometry{papergeometry}
\loadgeometry{papergeometry}

\renewcommand{\baselinestretch}{1.5}
\setlength{\columnsep}{2em} % 两栏间距
%\setlength{\columnseprule}{1} % 两栏间分割线宽度,默认值为0。
\setlength{\parskip}{0em} % 段落间距
\setlength{\lineskip}{1.5pt} % 行间距
\setlength{\parindent}{2em} % 段落缩进为2个基本字体。
\setlength{\floatsep}{3pt plus 3pt minus 2pt}      % 图形之间或图形与正文之间的距离
\setlength{\abovecaptionskip}{10pt plus 1pt minus 1pt} % 图形中的图与标题之间的距离
\setlength{\belowcaptionskip}{3pt plus 1pt minus 2pt} % 表格中的表与标题之间的距离


% 文章标题
\def\VARtitle{}
\renewcommand{\title}[1]{
    \def\VARtitle{#1}
}

% 文章作者
\def\VARauthor{}
\renewcommand{\author}[1]{
    \def\VARauthor{#1}
}

% 作者单位
\def\VARaddress{}
\newcommand{\address}[1]{
    \def\VARaddress{#1}
}

% 关键词
\def\VARkeywords{}
\newcommand{\keywords}[1]{
    \def\VARkeywords{#1}
}

% 中文摘要
\def\VARcnabstract{}
\newcommand{\cnabstract}[1]{
    \def\VARcnabstract{#1}
}

%%%%%%%%%% header & footer %%%%%%%%%%
% 页眉页脚

\usepackage{fancyhdr}
\fancypagestyle{frontmatter}{
    \renewcommand{\headrulewidth}{0pt}
    \renewcommand{\headrulewidth}{0pt}
    \fancyhf{}
    \fancyfoot[C]{\thepage}

}
\fancypagestyle{mainmatter}{
    \fancyhead{}
    \fancyfoot{}
    \if@twoside
        \fancyhead[CO]{\zihao{-5}\songti
            \VARtitle \vspace{1.5mm}
        }
        \fancyhead[CE]{\zihao{-5}\songti\leftmark\vspace{1.5mm}}
    \else
        \fancyhead[C]{\zihao{-5}\songti
            \ifthenelse{\isodd{\value{page}}}
                { \VARtitle }
                {\leftmark}
            \vspace{1.5mm}
        }
    \fi
    \fancyfoot[C]{\zihao{5}\thepage}

    \renewcommand{\headrulewidth}{0.5bp} % 页眉线宽度
}

\fancypagestyle{plain}{
    \pagestyle{fancy}
}

\renewcommand{\maketitle}{%
    \pagestyle{frontmatter}
    \begin{strip}
    \articletitle
    \end{strip}
}

%%%%%%%%%% title %%%%%%%%%%
% 标题

% 汉化
\renewcommand{\contentsname}{目\qquad 录}
\renewcommand\listfigurename{插\ 图\ 目\ 录}
\renewcommand\listtablename{表\ 格\ 目\ 录}
\renewcommand{\figurename}{图}
\renewcommand{\tablename}{表}
% 格式
% 章标题小三号团体居中
%\ctexset { chapter = {
    %format={\zihao{-3}\heiti},
    %nameformat={},
    %aftername={\quad},
    %titleformat={},
    %name={,},
    %number={\chinese{chapter}},
    %beforeskip={-.5\baselineskip},
    %afterskip={\baselineskip},
%}}
% 节标题小四号黑体居左
\ctexset { section = {
    format={\zihao{-4}\heiti},
    aftername={\quad},
    beforeskip={.5\baselineskip},
    afterskip={.5\baselineskip},
}}
% 条标题五号黑体居左
\ctexset { subsection = {
    format={\zihao{5}\heiti},
    aftername={\quad},
    beforeskip={.5\baselineskip},
    afterskip={.5\baselineskip},
}}
% 段标题五号宋体居左
\ctexset { subsubsection = {
    format={\zihao{5}\songti},
    aftername={\quad},
    beforeskip={.5\baselineskip},
    afterskip={.5\baselineskip},
}}

% 网址排版
\usepackage[colorlinks,linkcolor=black,anchorcolor=black,citecolor=black,CJKbookmarks=True]{hyperref}

% 算法排版
\usepackage{algorithm}
\usepackage{algorithmic}
\floatname{algorithm}{算法}
\renewcommand{\algorithmicrequire}{ {输入:}}
\renewcommand{\algorithmicensure}{ {输出:}}
%\renewcommand{\algorithmicreturn}{ {返回}}

% 定义定理环境
%\newtheoremstyle{mystyle}{3pt}{3pt}{\kaishu}{0cm}{\heiti}{\heiti}{}{1em}{}
%\theoremstyle{mystyle}
\newtheorem{definition}{\hspace{2em}定义}
\newtheorem{theorem}{\hspace{2em}定理}
\newtheorem{corollary}{\hspace{2em}推论}
\newtheorem{remark}[definition]{\hspace{2em}注}
\newtheorem{proposition}[definition]{\hspace{2em}命题}
\newtheorem{axiom}[definition]{\hspace{2em}公理}
\newtheorem{lemma}[definition]{\hspace{2em}引理}

% 解决pandoc转换后,tightlist编译出错问题
\newcommand{\tightlist}{%
  \setlength{\itemsep}{0pt}\setlength{\parskip}{0pt}}

%%%%%%%%%% abstract %%%%%%%%%%
% 摘要

% 中文摘要
\newenvironment{cabstract}{%
    \vspace*{2bp}
    {\zihao{-5}\heiti 摘要:}
    }{%

        {\zihao{-5}\heiti 关键词:}{\zihao{-5}\songti \VARkeywords}
}

%%%%%%%%%% reference %%%%%%%%%%
% 参考文献
\usepackage[authoryear,square,sort&compress]{natbib}
\bibpunct{[}{]}{,}{n}{}{}
\setlength{\bibsep}{0pt}
\usepackage{hypernat}
\newcommand{\upcite}[1]{\textsuperscript{\cite{#1}}}
\renewcommand{\citet}[1]{\citeauthor{#1} (\citeyear{#1})}
\renewcommand{\citep}[1]{(\citeauthor{#1}, \citeyear{#1})}
\renewcommand\bibname{参\ 考\ 文\ 献} % for report, book
%\renewcommand\refname{参\ 考\ 文\ 献}  % for article
\bibliographystyle{unsrtnat}

% 双栏设置
% 分栏不用另起一页,同一页即可以有双栏也可以有单栏
% 双栏最后一页上左右基本同高
% 如果需要插入单栏内容,只需要把单栏内容放在如下环境中即可:
% \begin{strip}
%   单栏内容
% \end{strip}
% 如果双栏最后一页先排滿左栏,然后再排右栏,需要使用命令
% \raggedend
% 恢复双栏平衡排列的命令为:
% \flushend
\usepackage{flushend,cuted}

% 单、双面设置
% 选用towside格式(book类文档默认选项),奇数面码打印在页面右边,偶数页码打印在页面左边。每章开始位置默认值openright,书的每个新的一单总是从奇数页开始,可能产生空白的偶数页;openany则表示每章从新的一页开始,不管这一页是奇数页还是偶数页。
% 选用oneside格式(article类文档和report类文档默认选项),所有页码的打印位置相同。


%%%%%%%%%% table %%%%%%%%%%
% 表格

% 长表格
\usepackage{longtable}

% 表格中的行合并
\usepackage{multirow}

% 重定义table/tabular/tabularx环境,使表格内为5号字
% TODO(huxuan): 支持 longtable
\let\oldtable\table
\let\endoldtable\endtable
\renewenvironment{table}[1][h!]
{\renewcommand{\arraystretch}{1.5}
\oldtable[#1]\zihao{5}}
{\renewcommand{\arraystretch}{1}
\endoldtable}

\let\oldtabular\tabular
\let\endoldtabular\endtabular
\renewenvironment{tabular}[1][h!]
{\renewcommand{\arraystretch}{1.5}
\oldtabular[#1]\zihao{5}}
{\renewcommand{\arraystretch}{1}
\endoldtabular}

\usepackage{tabularx}
\let\oldtabularx\tabularx
\let\endoldtabularx\endtabularx
\renewenvironment{tabularx}[2]
{\renewcommand{\arraystretch}{1.5}
\zihao{5}\oldtabularx{#1}{#2}}
{\renewcommand{\arraystretch}{1}
\endoldtabularx}


\usepackage{array}


%\footnote[0]{\scriptsize{\receivedate, \modifydate.
%\\
%\mbox{\;}\hspace{2\ccwd}\doino		
 %\\ \authorsinfo}}
%\vspace{-1em}
%

%%%%%%%%%%%% Article Title %%%%%%%%%%%%%%

\newcommand{\articletitle}{
    \begin{spacing}{1.2}
    \begin{center}
        \begin{minipage}[c]{16cm}
        %\vskip 0bp
        \centerline{\zihao{3}\ziju{0.0}\heiti \VARtitle }
        \vskip 20bp
        \centerline{\zihao{5}\ziju{0.0}\songti \VARauthor }
        \vskip 5bp
        \centerline{\zihao{-5}\ziju{0.0}\songti \VARaddress }
        \vskip 10bp
        \begin{cabstract}
            {\zihao{-5}\songti \VARcnabstract}
        \end{cabstract}
        \end{minipage}
    \end{center}
    \end{spacing}
    %\vskip 10bp
}

\begin{document}

\title{正例与未标注学习综述}
\author{苏江文}
\address{(福建亿榕信息技术有限公司~平台业务部, 福州~350003)}
\keywords{正例与未标注学习;}
\cnabstract{在机器学习和数据挖掘中,正例与未标注学习正在成为研究关注的热点,该学习模式适用于多种特定场合,产生了许多相应的分类器。}

\maketitle

% 正文页眉页脚样式
\pagestyle{mainmatter}


% 正文
\section{引言}
% !Mode:: "TeX:UTF-8"

文本分析处理的对象是包含大量文本的语料库,无法人工一一翻阅的情况下,通过机器学习手段自动构建能表达语料库内容的主题概览视图,需要建立文本的表示模型。本研究从文本关键词共现关联的角度,探索通过关联词向量空间模型(CTVSM - Co-occurrence Term Vector Space Model)表达语料库的主题聚集、热点事件发现等。


\section{经典算法}
% !Mode:: "TeX:UTF-8"

\subsection{Biased SVM}
% !Mode:: "TeX:UTF-8"

\citet{1250918}提出Biased SVM算法\cite{1250918}:训练样本集为$\{(x_1,y_1),(x_2,y_2),\ldots,(x_n,y_n) \}$,其中$x_i$是输入向量,$y_i$是类别标记,$y_i \in \{1,-1\}$。假设前$k-1$个样本是正例样本(标记为1),则剩下的是未标记样本,我们将它们全部标记为负例样本(标记为-1)。




\subsection{S-EM}
% !Mode:: "TeX:UTF-8"

\citet{liu2002partially}提出的S-EM算法\cite{liu2002partially}是基于朴素贝叶斯(Navie Bayesian)分类和期望最大(EM)算法,算法细节来源于论文《Partially Supervised Classification of Text Documents》,针对只拥有可信正例样本(正例集合P)和未标记样本(混合集合M),没有可信负例样本情况,基于"间谍"技术、朴素贝叶斯和EM算法。

首先把所有未标记样本都认为是负例样本,学习得到一个贝叶斯分类器NB-C。用这个分类器NB-C分类未标注样本,超过阈值的被认为是正例样本,迭代创建新的NB-C直到稳定。

\subsubsection{贝叶斯文本分类}

给定训练文本集$D$,每一篇文本被看作是有序的词列表。我们用$w_{d_i,k}$表示文本$d_i$中位置$k$的词,每个词来自于词汇表$V = < w_1,w_2,\ldots,w_{\left | v \right |}>$。我们同时还有一个预定义类别集合$C = \{ c_1,c_2,\ldots,c_{\left | C \right |} \}$(二分类的情况下,$C = \{c_1, c_2\}$)。为了实现分类,需要计算后验概率$Pr[c_j \mid
d_i]$,其中$c_j$是类别,$d_i$是文本。基于贝叶斯概率和多项式模型,可以得到类别概率$Pr[c_j]$(公式 \ref{pr_cj})和拉普加斯平滑后的已知类别后词的条件概率$Pr[w_t \mid c_j$(公式 \ref{pr_wt_cj})。

\begin{equation}
Pr\left [c_j \right ] = \sum\limits_iPr \left [ c_j \mid d_i \right ] / \left | D \right |
\label{pr_cj}
\end{equation}

\begin{equation}
Pr\left [w_t \mid c_j \right ] = \frac{1 + \sum\limits_{i=1}^{\left | D \right |}N(w_t,d_i)P(c_j \mid d_i)}{\left | V \right | + \sum\limits_{s=1}^{\left | V \right |}{\sum\limits_{i=1}^{\left | D \right |}N(w_s,d_i)P(c_j \mid d_i)}}
\label{pr_wt_cj}
\end{equation}

其中,$N(w_t, d_i)$是词$w_t$出现在文本$d_i$中总次数,公式 \ref{pr_cj}中的$Pr[c_j \mid d_i] \in \{0,1\}$依赖于文本的类别标注。最后,假设词的概率是独立于类别的,我们得到公式 \ref{pr_cj_di}。

\begin{equation}
Pr\left [ c_j \mid d_i \right ] = \frac{Pr \left [ c_j \right]\prod\limits_{k=1}^{\left | d_i \right |}Pr \left [ w_{d_i,k} \mid c_j\right ]}{\sum\limits_{r=1}^{\left | C \right |} Pr \left [ C_r \right ]\prod\limits_{k=1}^{\left | d_i \right |} P(w_{d,k} \mid c_r)}
\label{pr_cj_di}
\end{equation}

在朴素贝叶斯分类器中,文本类别由最高$Pr[c_j \mid d_i]$的类别确定。

\subsubsection{EM算法}

\subsubsection{步骤一:重新初始化(Reinitialization)}

\paragraph{应用EM算法(I-EM)}
初始化时,将正例样本集合$P$中所有样本标记为$c_1$类,即$Pr[c_1 \mid d_i]$ = 1, $Pr[c_2 \mid d_i]$ = 0,将未标记集合$M$中所有样本标记为$c_2$类,即$Pr[c_2 \mid d_j]$ = 1, $Pr[c_1 \mid d_j]$ = 0。此时构建了第一个朴素贝叶斯分类器NB-C。这个分类器被用于分类未标记集合$M$,使用公式 \ref{pr_cj_di}计算未标记集合$M$中每个文本的后验概率$Pr[c_1 \mid
d_j]$,将此概率赋给$d_j$作为它的新的概率分类标记。所有正例文本的分类概率保持不变,即$Pr[c_1 \mid d_i]$ = 1。

未标识集合$M$中所有$Pr[c_1 \mid d_j]$被修正后,同时还可计算出$Pr[w_t \mid c_k]$(公式 \ref{pr_wt_cj})和$Pr[c_k]$(公式 \ref{pr_cj}),由此可以构建出新的NB-C分类器,下一迭代开始,直到$EM$收敛。整个过程被称为I-EM(initial EM),详见算法 \ref{alg:iem}。


\begin{algorithm}[htb]
\caption{The I-EM algorithm with naive Bayesian classifier}
\label{alg:iem}
\begin{algorithmic}[1]
\REQUIRE 混合集合$M$,正例集合$P$
\STATE Build an initial naive Bayesian classifier NB-C using the document sets $M$ and $P$;
\WHILE {classifier parameters change}
\FOR {each $d_j \in M$}
\STATE Compute $Pr[c_1 \mid d_j]$ using the current NB-C;
\STATE // $Pr[c_2 \mid d_j] = 1 - Pr[c_1 \mid d_j]$
\STATE Update $Pr[w_t \mid c_1]$ and $Pr[c_1]$ given the probabilistically assigned class for $d_j$($Pr[c_1 \mid d_j]$) and $P$ (a new NB-C is being built in the process);
\ENDFOR
\ENDWHILE
\RETURN 
\end{algorithmic}
\end{algorithm}
	未标记集合$M$中每个文本$d_j$最终的概率分类标记可以被用来从混合集合中识别出正类样本。经验显示,此方法的效果好于原始朴素贝叶斯分类方法。

对于人工可以明显区分正负类的情况,I-EM算法可以有很好的效果。但对于人工较难区分正负类的情况,则效果欠佳,这是由于算法初始时强烈地偏向正类样本,因些还需要以下算法加以提升。

\paragraph{将“间谍”文本加入混合集合中}

经过I-EM后,我们有了一个很好的机会来识别混合集合M中的最近似负类样本。关键方法是从正类集合$P$中选取“间谍”样本到混合集合$M$中。从正类集合$P$中随机选取$x\%$样本(经验数据是$10\%$)组成间谍集合$S$,将其加入到混合集合$M$中。

对集合$M + S$应用上面的I-EM算法后,间谍样本的概率分类标记被用于决定哪 些样本是最近似负类样本。设定一个阈值$t$,混合集合$M$中样本概率分类标记小于阈值$t$的样本集合是最近似负类样本集合$N$,混合集合中样本概率标记大于阈值$t$的样本集合(不包括间谍样本)是未标记样本集合$U$。具体算法流程图见算法 \ref{alg:ident_likely_negative}。


\begin{algorithm}[htb]
\caption{Identifying likely negative documents}
\label{alg:ident_likely_negative}
\begin{algorithmic}[1]
\STATE $N = U = \phi$;
\STATE $S = sample(P, s\%)$;
\STATE $MS = M \cup S$;
\STATE $P = P - S$;
\STATE Assign every document $d_i$ in $P$ the class $c_1$;
\STATE Assign every document $d_j$ in $MS$ the class $c_2$;
\STATE Run I-EM($MS$, $P$);
\STATE Classify each document $d_j$ in $MS$;
\STATE Determine the probability threshold t using $S$;
\FOR {each document $d_j$ in $M$}
\IF {its probability $Pr{c_1 \mid d_j] < t}$ }
\STATE $N = N \cup \{d_j\}$;
\ELSE
\STATE $U = U \cup \{d_j\}$;
\ENDIF
\ENDFOR
\RETURN 
\end{algorithmic}
\end{algorithm}

阈值$t$的计算:无噪声的情况下,取间谍集合中最小的概率分类标记,即$t = min\{Pr[c_1 \mid s_1],Pr[c_1 \mid s_2],\ldots,Pr[c_1 \mid s_k]\}$。但通常情况下都有噪声样本,这些噪声样本的概率分类标记可能为0或者远小于其它间谍样本。因此,可以定义一个噪声级别参数 $l$,间谍样本集合中$l\%$样本的概率分类标记小于阈值$t$。噪声级别$l$可以取值5,10,15,或20,经验值是$15\%$。


\subsubsection{步骤二:用样本集合$P$、$N$、$U$构建最终分类器(S-EM)}

The step builds the final classifier. The $EM$ algorithm is again employed, with the document sets, $P$, $N$, and $U$. This step is carried out as follows 算法 \ref{alg:sem}.

\begin{algorithm}[htb]
\caption{S-EM algorithm}
\label{alg:sem}
\begin{algorithmic}[1]
    \STATE Put all the spy documents $S$ back to the positive set $P$.
    \STATE Assign every document in the positive set $P$ the fixed class label $c_1$, $Pr[c_1 \mid d_i] = 1$, which will not chage in each iteration of $EM$.
    \STATE Assign each document $d_j$ in the likely negative set $N$ the initial class $c_2$, i.e., $P4[c_2 \mid \d_j] = 1$, which changes with each iteration of $EM$.
    \STATE Each document $d_k$ in the unlabeled set $U$ is not assigned any label initially. At the end of the first iteration of $EM$, it will be assigned a probabilistic label, $Pr[c_1 \mid d_k]$. In subsequent iterations, the set $U$ will participate in $EM$ with its newly assigned probabilistic classes, e.g., $(Pr[c_1 \mid d_k])$.
    \STATE Run the $EM$ algorithm using the document sets $P$, $N$, and $U$ until it conerges.
    \RETURN the inal classifier.
\end{algorithmic}
\end{algorithm}



\subsection{MLEL}
% !Mode:: "TeX:UTF-8"

\citet{AAAI113583}提出MLEL算法\cite{AAAI113583},首先使用一个启发式方法(Heuristic Method)按可信度(Confidence)生成多级样本(Multi-level examples),样本的数量和质量对训练一个高质量的分类器非常重要。难以同时保证准确率和召回率,必须在两者之间选择权衡。其次,使用加权支持向量机(WSVM)来不同对待多级训练样本。共分为五级:$GP$(Golden Positives)、$PP$(Potential Positives)、$SN$(String Negatives)、$RN$(Reliable Negatives)、$PN$(Potential
Negatives)。正例度$PD$(Positive Degree)被用来裁决一个未标注样本是否正例样本。

\begin{algorithm}[htb]
\caption{MLEL($P,U$)}
\label{alg:MLEL}
\begin{algorithmic}[1]
\REQUIRE positive documents $P$, unlabeled document $U$
\ENSURE a text classifier
\STATE Obtain positive feature set ($PF$) and word positive
\STATE degree($PD_{word}$) for each feature using Positive Feature
\STATE Selection algorithm.
\STATE Use Multi-level Example Generation algorithm to
\STATE obtain $GP, PP, SN, RN$ and $PN$.
\STATE Train text classifier using WSVM.
\RETURN 
\end{algorithmic}
\end{algorithm}

\subsubsection{Positive Feature Selection}
正例特征使用可以表达正例样本并能与负例样本区分开的词条。用两个统计标准量$Specialty$和$Popularity$来度量一个词条是否正例特征。
当一个词条在正例样本集合$P$中出现的频率超过在混合集合中出的频率时,该词条更倾向地是一个正例特征。如下公式,当词条的$Specialty > 0.5$时,更倾向于是一个正例特征。

\begin{equation}
Specialty(w) = f(w,P)/(f(w,P) + f(w,U))
\end{equation}

假设集合$P$中两个词条具有相同的出现频率,其中在集合$P$中更多样本中出现的那个词条,相对另一个更倾向于正例特征。用信息熵$Ent(w,P)$来描述集合P中词条$w$的分布情况。

\begin{equation}
Popularity(w) = Ent(w,P) / Z
\end{equation}

$Z$是信息熵$Ent(w,P)$的最大值,可以取$log(n_p)$,$n_p$是集合P中样本总数。

\begin{equation}
Ent(w,P) = -\sum\limits_{i=1}^{n_p}NProb(d_i \mid w)log(NProb(d_i \mid w))
\end{equation}

\begin{equation}
NProb(d_i \mid w) = \frac{Prob(d_i \mid w) / l_i}{\sum\limits_{j=1}^{n_p}Prob(d_j \mid w) / l_j} 
\end{equation}

\begin{equation}
Prob(d_i \mid w) = f(w,d_i)/f(w,P) 
\end{equation}

\begin{equation}
l_i = \sum\limits_{w \in d_i}f(w,d_i)
\end{equation}

\begin{equation}
PD_{word}(w) = Specialty(w) + Popularity(w)
\end{equation}

判别条件满足$Popularity(w)$ > $\alpha$,$Specialty(w)$ > $\beta$,$PD_{word}(w)$ > $\gamma$时,词条w是一个正例特征。$\alpha$、$\beta$、$\gamma$分别是$Popularity$、$Specialty$、$PD_{word}$的判定阈值,经验值可以取实验中的平均值。

\subsubsection{Multi-Level Example Generation}

\begin{itemize}
\item Document Positive Degree  

样本正例度越大,该样本越倾向于是一个正例样本。

\begin{equation}
PD_{doc}(d_i) = \frac{\sum\limits_{w \in PF, w \in d_i}PD_{word}(w)}{log(l_i)}
\end{equation}

\item Multi-Level Positives Acquisition

已有的正例训练样本集合P中的样本被称为$GP$(Golden Positives)。
样本$PD_{doc}$大于集合P中所有样本$PD_{doc}$的平均值的,被称为$PP$(Potential Positives)。

\begin{equation}
PD_{doc}(d_x) > \overline{PD(P)} 
\end{equation}

\item Multi-Level Negatives Acquisition

样本$PD_{doc}$ = 0,被称为$SN$(Strong Negatives)。通常是不包含任何正向特征的样本。
样本$PD_{doc}$小等于集合U中所有样本$PD_{doc}$的平均值的,被称为$RN$(Reliable Negatives)。

\begin{equation}
0 < PD_{doc}(d_x) \leq \overline{PD(U)}
\end{equation}

集合M中刨除$PP$、$SN$、$RN$剩余的样本被称为$PN$(Potential Negatives)。经验表明,PN对训练分类器同样有用。

\end{itemize}

\subsubsection{Multi-Level Example Based Learning}

使用加权支持向量机WSVM基于多级样本训练分类器,给样本分别赋予不同的权值。

最小化: $\frac{1}{2}\left \| W \right \| ^2 + 
        c_{+}^{'}\sum_{i \in GP}\xi_i +
        c_{+}^{''}\sum_{i \in PP}\xi_i +
        c_{-}^{'}\sum_{i \in SN}\xi_i +
        c_{-}^{''}\sum_{i \in RN}\xi_i +
        c_{-}^{'''}\sum_{i \in PN}\xi_i 
        $

优化目标: $y_i(w^Tx_i + b) \geq 1 - \xi_i (i = 1,2, \ldots, n)$

$\xi_i$是松弛变量(slack variable),用于允许误分类部分训练样本。$c_{+}$$c_{-}$等变量分别是$GP$、$PP$、$SN$、$RN$、$PN$的惩罚因子(Penalty Factor),用于调整不同级别样本可信度造成的影响。通常正例样本数量远远低于 未标注样本集合中负例样本数量。




\subsection{Bagging SVM}
% !Mode:: "TeX:UTF-8"

\citet{Mordelet2014201}提出了bagging SVM算法\cite{Mordelet2014201}解决正例与未标识问题。

\begin{algorithm}[htb]
\caption{Inductive bagging PU learning}
\label{alg:bagging_SVM}
\begin{algorithmic}[1]
\REQUIRE $P,U,K = $ size of bootstrap samples, $T = $ number of bootstraps.
\ENSURE a function $f: X \to R$
\FOR{ $t = 1$ to $T$ }
    \STATE Draw a subsample $U_t$ of size $K$ from $U$.
    \STATE Train a classifier $f_t$ to discriminate $P$ against $U_t$.
\ENDFOR
\RETURN $f = \frac{1}{T}\sum\limits_{t=1}^Tf_t$
\end{algorithmic}
\end{algorithm}







\section{实验}

\section{结论}

\section{相关工作}


% 参考文献
\onecolumn
\bibliographystyle{unsrtnat}
\bibliography{ref/pulearning}

\end{document}

